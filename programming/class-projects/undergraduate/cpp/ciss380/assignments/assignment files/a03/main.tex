\input{myassignmentpreamble.tex}
\renewcommand\AUTHOR{David Campbell}
\renewcommand\COURSENAME{Computer Graphics}
\renewcommand\COURSESHORTNAME{Computer Graphics}
\renewcommand\COURSENUMBER{CISS380}
\renewcommand\TITLE{Assignment 3}
\begin{document}

\newcommand\vv[2]{ \langle #1, #2 \rangle }
\topmatter

{\bf Objectives}

This assignment involves vector and basic matrix computations.

You must submit your work as a printed pdf document.
You must generate the pdf using \LaTeX.
Most of you have already done a lot of \LaTeX.
The important thing is that you must use the latest 
Fedora virtual machine because I've reorganized a lot of my 
\LaTeX\ libraries and setup files on the newest 
Fedora virtual machine to make using \LaTeX\ easier.

The following is only for those of you who have not taken classes 
with me where there is mathematical writing using
\LaTeX\:
\begin{enumerate}
\item For each assignment, you will be given a zipped file.
      Unzip the file in your virtual machine.
\item The unzipped folder contains a \verb!makefile!.
      When you run the \verb!makefile! (if you're using the GUI, you just
      double-click on the \verb!makefile!; if you're using the  
      Linux bash shell, you type \verb!make!), a pdf document 
      will be (re)generated.
      The name of this file is \verb!main.pdf!.
      You're looking at it right now.
\item In your unzipped folder, you will see the main \LaTeX\ file:
      \verb!main.tex!.
      The only thing you need to modify for this file is to change the 
      \verb!\AUTHOR! to your name.
      Currently \verb!\AUTHOR! is set to \lq\lq Dr. Yihsiang Liow''.
      This is near the top of \verb!main.tex! (usually the second line.)
      When you open the file you will see what I mean.
      After changing my name to yours, run the \verb!makefile! and
      your name will appear in the bottom left of every page of this 
      pdf document.
\item You will need to edit the \LaTeX\ files 
      \verb!q01.tex!,
      \verb!q02.tex!,
      \verb!q03.tex!, etc.
      The file \verb!q01.tex! is the \LaTeX\ file for 
      your answer to question Q1. 
      When you are ready to typeset your answer for Q1,
      you edit \verb!q01.tex! and run the \verb!makefile!
      and view the updated \verb!main.pdf!.
      The new \verb!main.pdf! will contain your work for Q1.
      When you run the \verb!makefile!, \verb!main.tex! will collect up
      all your answers in \verb!q01.tex!, \verb!q02.tex!, etc. to build
      \verb!main.pdf!.
\item Print your final \verb!main.pdf!, staple the printouts, and submit
      them to me. 
\item Sloppy work, regardless of correctness of answer, will get a 0.
      For instance if your work is not stapled or not stapled in the 
      right order, you will get a 0 immediately.
\end{enumerate}

\newpage

{\bf VECTORS}

Here's the story so far: You've been learning about real numbers
for a long time. Here's a real number:
\[
42
\]
and here's another
\[
3.1415
\]
You can add, subtract, multiply, and divide real numbers 
(well ... hang on there ... not quite ... you can't divide by zero, right?)

Then you learned about 
2-dimensional (2-d) vectors which 
are slightly more complicated since there are
two real numbers in each vector.
Here's a 2-d vector:
\[
\vv{42}{3.1419}
\]
For 2-d vectors, you can add, subtract, multiply-by-real numbers, 
and do dot products between them:
\begin{align*}
\vv{1}{2} + \vv{3}{4} &= \vv{1+3}{2+4} \\
\vv{1}{3} - \vv{2}{-4} &= \vv{1-2}{3-(-4)} \\
5 \cdot \vv{1}{3} &= \vv{5 \cdot 1}{5 \cdot 3} \\
\vv{3}{-2} \cdot \vv{4}{3} &= 3\cdot 4 + (-2) \cdot 3 \\
\end{align*}
and remember that the important thing about the dot product is this:
\[
\vec{u} \cdot \vec{v} = \| u \| \| v \| \cos \theta
\]
which ties up the two vectors with the angle between them.

Here's a bunch of drill problems:

\newpage

Q1. Compute the following
\begin{enumerate}
\item[(a)] $ \vv{3}{1/2} + \vv{-5/2}{3/4} $
\item[(b)] $\vv{3/2}{1/2} + \vv{5/3}{4} + \vv{3}{0}$
\item[(c)] $5\vv{3}{3/2} - 7\vv{-5}{4}$
\item[(d)] $\frac{1}{2} \vv{3}{1/2} + 7 \vv{-2}{3} - 3 \vv{0}{-14/3}$
\item[(e)] $\frac{1}{2} \vv{3}{1/2} \cdot 7 \vv{-2}{3}$
\item[(f)] What is $\cos\theta$ where $\theta$ is the angle between
$\vv{3}{1/2}$ and $\vv{2}{3}$
\item[(g)] What is $\cos\theta$ where $\theta$ is the angle between
$135235\cdot \vv{3}{1/2}$ and $234545621\cdot \vv{2}{3}$
[HINT: The big multiples are red herrings.]
\item[(h)] What is $\theta$ (in radians) where $\theta$ is the angle between
$\vv{7}{-3.4}$ and $\vv{4.2}{5.6}$.
You can use your calculator to give an approximation.
\item[(i)] What is $\theta$ (in degrees) where $\theta$ is the angle between
$\vv{7.2}{7.2}$ and $\vv{-5.66}{5.66}$.
You must give an exact value.
\end{enumerate}

\SOLUTION

(a) 
\[
\vv{3}{1/2} + \vv{-5/2}{3/4}
= \vv{3 - 5/2}{1/2 + 3/4} 
= \vv{1/2}{5/4}
\]
ANSWER: \boxed{ \vv{1/2}{5/4} }

(b) 
\[
\vv{3/2}{1/2} + \vv{5/3}{4} + \vv{3}{0}
= \vv{3/2 + 5/3 + 3}{1/2 + 4 + 0}
= \vv{37/6}{9/2}
\]
ANSWER: \boxed{\vv{37/6}{9/2}}

(c) 
\begin{align*}
5\vv{3}{3/2} - 7\vv{-5}{4} &= \vv{5  \cdot  3}{5  \cdot  3/2} - \vv{7  \cdot  -5}{7  \cdot  4}\\
&= \vv{15}{15/2} - \vv{-35}{28}\\
&= \vv{15 -(-35)}{15/2 - 28}\\
&= \vv{50}{-41/2}\\
\end{align*}
ANSWER: \boxed{\vv{50}{-41/2}}

(d)
\begin{align*}
1/2\vv{3}{1/2} + 7\vv{-2}{3} - 3\vv{0}{-14/3} &= \vv{1/2 \cdot 3}{1/2 \cdot 1/2} + \vv{7 \cdot -2}{7 \cdot 3} - \vv{3 \cdot 0}{3 \cdot -14/3}\\
&= \vv{3/2}{1/4} + \vv{-14}{21} - \vv{0}{-14}\\
&= \vv{3/2 + -14 - 0}{1/4 + 21 -(-14)}\\
&= \vv{-25/2}{141/4}
\end{align*}
ANSWER: \boxed{\vv{-25/2}{141/4}}

(e) 
\begin{align*}
1/2\vv{3}{1/2} \cdot 7\vv{-2}{3} &= \vv{1/2 \cdot 3}{1/2 \cdot 1/2} \cdot \vv{7 \cdot -2}{7 \cdot 3}\\
&= \vv{3/2}{1/4} \cdot \vv{-14}{21}\\
&= (3/2 \cdot -14) + (1/4 \cdot 21)\\
&= -21 + 21/4\\
&= -63/4
\end{align*}
ANSWER: \boxed{-63/4}

(f) We have
\begin{align*}
\vv{3}{1/2} \cdot \vv{2}{3} &= \| \vv{3}{1/2} \|  
                               \cdot 
                               \| \vv{2}{3} \| 
                               \cdot
                               \cos \theta
\\
\THEREFORE \cos\theta     &= \frac {\vv{3}{1/2} \cdot \vv{2}{3}} 
                               {\| \vv{3}{1/2} \|  \cdot \| \vv{2}{3} \| }
\\
                          &= \frac {(3)(2) + (1/2)(3)}
                             {\sqrt{3^2 + 1/2^2} \cdot \sqrt{2^2 + 3^2}}
\\
                          &= \frac {15/2}
                             {\sqrt{37/4} \cdot \sqrt{13}}
\\
                          &= \frac{15/2}{\sqrt{481/4}}
\end{align*}


ANSWER: \boxed{\frac{15/2}{\sqrt{481/4}}}

(g)
\begin{align*}
\cos\theta &= \frac{135235 \cdot \vv{3}{1/2} \cdot 234545621 \cdot \vv{2}{3}}
{\sqrt{135235 \cdot 3^2 + 135235 \cdot 1/2^2} \cdot \sqrt{234545621 \cdot 2^2 + 234545621 \cdot 3^2}}\\
&= \frac{\vv{3}{1/2} \cdot \vv{2}{3}}{\sqrt{3^2 +1/2^2} \cdot \sqrt{2^2 + 3^2}}\\
&= \frac{(3 \cdot 2) +(1/2 \cdot 3)}{\sqrt{37/4 \cdot 13}}\\
&= \frac{15/2}{\sqrt{481/4}}
\end{align*} 
ANSWER: \boxed{\frac{15/2}{\sqrt{481/4}}}

(h) 
We have
\begin{align*}
\vv{7}{-3.4} \cdot \vv{4.2}{5.6} &= \| \vv{7}{-3.4} \|  
                                    \cdot 
                                    \| \vv{4.2}{5.6} \| 
                                    \cdot
                                    \cos \theta
\\
\THEREFORE \cos\theta    &= \frac {\vv{7}{-3.4} \cdot \vv{4.2}{5.6}} 
                             {\| \vv{7}{-3.4} \|  \cdot \| \vv{4.2}{5.6} \| }
\\
                         &= \frac {(7)(4.2) + (-3.4)(5.6)}
                            {\sqrt{7^2 + 3.4^2} \cdot \sqrt{4.2^2 + 5.6^2}}
\\
                         &\approx 0.1902 \hspace{0.5cm} \text{(up to 4 dec. pl.)}
\\
\THEREFORE \theta        &\approx \cos^{-1} 0.1902 
\\
                         &\approx 1.3794
\end{align*}

ANSWER: \boxed{1.3794}

(i) 
We have
\begin{align*}
\vv{7.2}{7.2} \cdot \vv{-5.66}{5.66} &= \| \vv{7.2}{7.2} \|  
                                    \cdot 
                                    \| \vv{-5.66}{5.66} \| 
                                    \cdot
                                    \cos \theta
\\
\THEREFORE \cos\theta    &= \frac {\vv{7.2}{7.2} \cdot \vv{-5.66}{5.66}} 
                             {\| \vv{7.2}{7.2} \|  \cdot \| \vv{-5.66}{5.66} \| }
\\
                         &= \frac {(7.2)(-5.66) + (7.2)(5.66)}
                            {\sqrt{7.2^2 + 7.2^2} \cdot \sqrt{-5.66^2 + -5.66^2}}
\\
                         &= 0 \\
\THEREFORE \theta        &= \cos^{-1} 0 
\\
                         &= 90
\end{align*}
ANSWER: \boxed{90 ^\circ}



\newpage
{\bf MATRICES}

To move on, we now consider blocks of numbers called matrices.
Here's a 2-by-2 matrix:
\[
\begin{bmatrix}
1 & 2 \\
3 & 4
\end{bmatrix}
\]
and here's a 3-by-5 matrix:
\[
\begin{bmatrix}
1 & 2 & 2.5 & -1.2 & 7 \\
0 & 4.2 & 2.5 & -1.2 & 6 \\
0 & 0 & 2.5 & -1.2 & 4.2 \\
\end{bmatrix}
\]
For that second big matrix, we say that the 
row size is 3 and the column size is 5.
A 2-d vector is just a 2-by-1 matrix:
\[
\vv{1}{-5} \longrightarrow
\begin{bmatrix}
1 \\
-5
\end{bmatrix}
\]
I won't explain here why we need matrices to help out with computer
graphics.
I'll explain in detail during lectures.
But they are extremely important.
In fact their importance is beyond computer
graphics. Without matrices, you won't have quantum mechanics.
I'm not kidding. 
(OK ... alright ... you don't care about quantum
mechanics ... {\it whatever!})
Right now I just want to show you how to add, subtract, 
multiply-by-real-number, and multiply a matrix with another.
So this section is more about following instructions (or algorithms).

If the row size is the same as the column size, the matrix
is called a square matrix (duh).
Something like this
\[
\begin{bmatrix}
1 \\
-5
\end{bmatrix}
\]
is called a column vector while something like this
\[
\begin{bmatrix}
1 & -5
\end{bmatrix}
\]
is called a row vector.

It's handy to refer to values by their positions.
So for the following matrix:
\[
\begin{bmatrix}
1 & 2 & 2.5 & -1.2 & 7 \\
0 & 4.2 & 2.5 & -1.2 & 6 \\
0 & 0 & 2.5 & -1.2 & 4.2 \\
\end{bmatrix}
\]
the entry (or value) at row 2, column 5 is 6, or we say that the 
(2, 5) entry/value is 6.
For matrices, the row and column numbering starts with 1 (not 0).

\newpage

Q2. Unless otherwise stated, all answers must be exact.

Let
\[
A
= 
\begin{bmatrix}
1 & 2 & 3 \\
4 & 5 & 6 \\
\end{bmatrix}
, 
\hspace{1cm}
B
= 
\begin{bmatrix}
0.1 & -0.2 \\
-1.2 & 2.4 \\
-1.2 & 2.4 \\
\end{bmatrix}
, 
\hspace{1cm}
C
= 
\begin{bmatrix}
\frac{2}{3} & \frac{4}{7} \\
\end{bmatrix}
, 
\hspace{1cm}
D
= 
\begin{bmatrix}
3.14159 \\
\end{bmatrix}
\]

(a) What is the row size of $A$?

(b) What is the column size of $A$?

(c) What is the row size of $B$?

(d) What is the column size of $B$?

(e) What is the row size of $C$?

(f) What is the column size of $C$?

(g) What is the row size of $D$?

(h) What is the column size of $D$?

(i) What is the $(1, 3)$ entry of $A$?

(j) What is the $(3, 1)$ entry of $B$?


\SOLUTION

(a)
ANSWER: \boxed{2}

(b)
ANSWER: \boxed{3}

(c)
ANSWER: \boxed{3}

(d)
ANSWER: \boxed{2}

(e)
ANSWER: \boxed{1}

(f)
ANSWER: \boxed{2}

(g)
ANSWER: \boxed{1}

(h)
ANSWER: \boxed{1}

(i)
ANSWER: \boxed{3}

(j)
ANSWER: \boxed{-1.2}



\newpage
{\bf MATRIX ADDITION, SUBTRACTION, AND SCALAR MULTIPLICATION}

Adding, subtracting, and multiplying-with-real-numbers is easy.
By the way multiplication-by-real-number for matrices
is also called scalar multiplication, just like for vectors.

For adding and subtracting two matrices, first of all they must have
the same size. If they have different sizes, then all bets are off
and the operation is not defined.
If they have the same sizes, then you just operate on the values
based on their position in the matrices.
For instance, here's a 3-by-2 addition:
\[
\begin{bmatrix}
a & b \\
c & d.\\
e & f \\
\end{bmatrix}
+
\begin{bmatrix}
A & B \\
C & D.\\
E & F \\
\end{bmatrix}
=
\begin{bmatrix}
a+A & b+B \\
c+C & d+D.\\
e+E & f+F \\
\end{bmatrix}
\]
and a 2-by-8 addition:
\[
\begin{bmatrix}
a & b & c & d\\
e & f & g & h \\
\end{bmatrix}
+
\begin{bmatrix}
A & B & C & D\\
E & F & G & H \\
\end{bmatrix}
=
\begin{bmatrix}
a+A & b+B & c+C & d+D\\
e+E & f+F & g+G & h+H \\
\end{bmatrix}
\]
And of course you can't add a 3-by-2 with a 2-by-8:
\[
\begin{bmatrix}
a & b \\
c & d.\\
e & f \\
\end{bmatrix}
+
\begin{bmatrix}
A & B & C & D\\
E & F & G & H \\
\end{bmatrix}
= \text{WHADDA?!?}
\]
Subtraction is similar: do it position-wise.
Here's a subtraction for 3-by-4 matrices:
\[
\begin{bmatrix}
a & b & c & d \\
e & f & g & h \\
i & j & k & l \\
\end{bmatrix}
-
\begin{bmatrix}
A & B & C & D\\
E & F & G & H \\
I & J & K & L \\
\end{bmatrix}
=
\begin{bmatrix}
a-A & b-B & c-C & d-D\\
e-E & f-F & g-G & h-H \\
i-I & j-J & k-K & l-L \\
\end{bmatrix}
\]
Multiply-by-real-number or scalar multiplication is easy too.
Here's a multiplication of a real number with a 2-by-2:
\[
r \cdot 
\begin{bmatrix}
a & b \\
c & d.\\
\end{bmatrix}
=
\begin{bmatrix}
r \cdot a & r \cdot b \\
r \cdot c & r \cdot d.\\
\end{bmatrix}
\]

\newpage

Q3. Now to exercise your brain for adding, subtracting, and scalar 
multiplication of matrices.
For those expressions where the operations do not make sense,
write ERROR.
Unless otherwise stated, all answers must be exact.


Let 
\[
A
= 
\begin{bmatrix}
1 & 2 & 3 \\
4 & 5 & 6 \\
\end{bmatrix}
, 
\hspace{1cm}
B
= 
\begin{bmatrix}
0.1 & -0.2 \\
-1.2 & 2.4 \\
-1.2 & 2.4 \\
\end{bmatrix}
, 
\hspace{1cm}
C
= 
\begin{bmatrix}
\frac{2}{3} & \frac{4}{7} \\
\end{bmatrix}
, 
\hspace{1cm}
D
= 
\begin{bmatrix}
3.14159 \\
\end{bmatrix}
\]
\[
E
= 
\begin{bmatrix}
0.33 & -1 & -2.5 \\
0.7 & 2.5 & 1.23
\end{bmatrix}
, 
\hspace{1cm}
F
= 
\begin{bmatrix}
2.7182 \\
\end{bmatrix}
, 
\hspace{1cm}
G
= 
\begin{bmatrix}
1.1 & -0.2 \\
1.2 & -0.4 \\
1.3 & -0.8 \\
\end{bmatrix}
, 
\hspace{1cm}
H
= 
\begin{bmatrix}
1 & 2 & -3 \\
3 & 2 & -1 \\
\end{bmatrix}
\]

(a) Compute $A + C$.

(b) Compute $H + A$.

(c) Compute $2B - 3G$.

(d) Compute $2B + H$.

(e) Compute $B + H$.

(f) Compute $2D + 3F$.

(g) Compute $M$ if $2A + 2M + E = 3E - 2H + A$.

(h) Compute $N$ if $2D + 2N + F = 3N - D + 3C$.


\SOLUTION

(a) 
$A$ has a size of 2-by-3 while $C$ has a size of 1-by-2.
$A$ and $C$ have different sizes.
Therefore $A + C$ is undefined.

ANSWER: \boxed{ERROR}

(b)
\begin{align*}
H + A &= \begin{bmatrix}
         1 & 2 & -3 \\
         3 & 2 & -1 
         \end{bmatrix}
         + 
         \begin{bmatrix}
         1 & 2 & 3 \\
         4 & 5 & 6 
         \end{bmatrix}
         \\
      &= \begin{bmatrix}
         1 + 1     & 2 + 2   & (-3) + 3 \\
         3 + 4     & 2 + 5   & (-1) + 6 
         \end{bmatrix}
         \\
      &= \begin{bmatrix}
         2 & 4 & 0 \\
         7 & 7 & 5  
         \end{bmatrix}
\end{align*}
ANSWER: 
\boxed{
\begin{bmatrix}
2 & 4 & 0 \\
7 & 7 & 5
\end{bmatrix}
}

(c)
\begin{align*}
2B - 3G &= 2 \cdot \begin{bmatrix}
0.1 & -0.2 \\
-1.2 & 2.4 \\
-1.2 & 2.4
\end{bmatrix}
-
3 \cdot \begin{bmatrix}
1.1 & -0.2 \\
1.2 & -0.4 \\
1.3 & -0.8
\end{bmatrix}\\
&= \begin{bmatrix}
2 \cdot 0.1 & 2 \cdot -0.2 \\
2 \cdot -1.2 & 2 \cdot 2.4 \\
2 \cdot -1.2 & 2 \cdot 2.4
\end{bmatrix}
-
\begin{bmatrix}
3 \cdot 1.1 & 3 \cdot -0.2 \\
3 \cdot 1.2 & 3 \cdot -0.4 \\
3 \cdot 1.3 & 3 \cdot -0.8
\end{bmatrix}\\
&= \begin{bmatrix}
0.2 & -0.4 \\
-2.4 & 4.8 \\
-2.4 & 4.8
\end{bmatrix}
-
\begin{bmatrix}
3.3 & -0.6 \\
3.6 & -1.2 \\
3.9 & -2.4
\end{bmatrix} \\
&=
\begin{bmatrix}
-3.1 & -2.9 \\
-6 & 6 \\
-6.2 & 7.2
\end{bmatrix}
\end{align*}
ANSWER: \boxed{\begin{bmatrix}
-3.1 & -2.9 \\
-6 & 6 \\
-6.2 & 7.2
\end{bmatrix}}

(d) ANSWER: \boxed{ERROR}

(e)
ANSWER: \boxed{ERROR}

(f)
\begin{align*}
2D + 3F &= 2 \cdot \begin{bmatrix}
3.14159
\end{bmatrix}
+
3 \cdot \begin{bmatrix}
2.7182
\end{bmatrix} \\
&= \begin{bmatrix}
2 \cdot 3.14159
\end{bmatrix}
+
\begin{bmatrix}
3 \cdot 2.7182
\end{bmatrix} \\
&= \begin{bmatrix}
6.28318
\end{bmatrix}
+
\begin{bmatrix}
8.1546
\end{bmatrix} \\
&= \begin{bmatrix}
14.43778
\end{bmatrix}
\end{align*}
ANSWER: \boxed{\begin{bmatrix}
14.43778
\end{bmatrix}}

(g) First we simplify algebraically.
We have
\begin{align*}
            2A + 2M + E &= 3E - 2H + A           \\
\THEREFORE  2M          &= (3E - 2H + A) - (2A + E) \\
                        &= 2E + (-2H) + (-A)    \\
\THEREFORE  M           &= E - H - A/2                   \tag{1} \\
\end{align*}
Now we substitute the matrix values of $A$, $E$, and $H$ into (1) to get
\begin{align*}
M &= \begin{bmatrix}
0.33 & -1 & -2.5 \\
0.7 & 2.5 & 1.23
\end{bmatrix}
-
\begin{bmatrix}
1 & 2 & -3 \\
3 & 2 & -1
\end{bmatrix}
-
\begin{bmatrix}
1/2 & 1 & 3/2 \\
2 & 5/2 & 3
\end{bmatrix} \\
  &= \begin{bmatrix}
-1.17 & -4 & -1 \\
-4.3 & -2 & -0.77
\end{bmatrix}
\end{align*}

ANSWER: \boxed{\begin{bmatrix}
-1.17 & -4 & -1 \\
-4.3 & -2 & -0.77
\end{bmatrix}}

(h)
ANSWER: \boxed{ERROR}



\newpage

{\bf MATRIX MULTIPLICATION}

Now let me show you how to multiply matrices.
This is the one that is a little harder.
So I'll look at specific sized matrices instead of talking in general.

You can multiply two matrices $M$ and $N$ only when the 
column size of $M$ is the same as the row size of $N$.
You'll see why in a bit.
Here's the multiplication of a 2-by-3 with a 3-by-4:
\[
\begin{bmatrix}
a & b & c \\
d & e & f 
\end{bmatrix}
\cdot
\begin{bmatrix}
A & B & C & D \\
E & F & G & H \\
I & J & K & L  
\end{bmatrix}
= ?
\]
First of all here are the sizes:
\[
(\text{2-by-3}) \cdot (\text{3-by-4}) = (\text{?-by-?})
\]
Notice that the column size of the first (i.e. 3)
matches the row size of the second (i.e. 3).
The resulting matrix will have a row size equal to the row size of the 
first matrix:
\[
(\text{\underline{2}-by-3}) \cdot (\text{3-by-4}) = (\text{\underline{2}-by-?})
\]
and the column size of the result is the column size of the second matrix:
\[
(\text{2-by-3}) \cdot (\text{3-by-\underline{4}}) = (\text{2-by-\underline{4}})
\]
So the product looks like this:
\[
\begin{bmatrix}
a & b & c \\
d & e & f 
\end{bmatrix}
\cdot
\begin{bmatrix}
A & B & C & D \\
E & F & G & H \\
I & J & K & L  
\end{bmatrix}
= 
\begin{bmatrix}
? & ? & ? & ? \\
? & ? & ? & ?
\end{bmatrix}
\]
Here's how you compute the result at row 1, column 1 on the right-hand-side:
From the left-hand-side,
you take row 1 of the 1st matrix and column 1 of the 2nd 
matrix and treat them like vectors and do a dot product.
\[
\begin{bmatrix}
\underline{a} & \underline{b} & \underline{c} \\
d & e & f 
\end{bmatrix}
\cdot
\begin{bmatrix}
\underline{A} & B & C & D \\
\underline{E} & F & G & H \\
\underline{I} & J & K & L  
\end{bmatrix}
= 
\begin{bmatrix}
aA + bE + cI & ? & ? & ? \\
? & ? & ? & ?
\end{bmatrix}
\]
This is the reason why the column size of the first matrix
must match the row size of the second.

Now moving right, we want to compute the row 1, column 2 value
of the result. You don't change row 1 of the first matrix,
but you move to the right on the second matrix and look at column 2
of that matrix. Do the dot product thing again and you get the
value at row 1, column 2 on the right:
\[
\begin{bmatrix}
\underline{a} & \underline{b} & \underline{c} \\
d & e & f 
\end{bmatrix}
\cdot
\begin{bmatrix}
A & \underline{B} & C & D \\
E & \underline{F} & G & H \\
I & \underline{J} & K & L  
\end{bmatrix}
= 
\begin{bmatrix}
? & aB + bF + cJ & ? & ? \\
? & ? & ? & ?
\end{bmatrix}
\]
and of course you should guess immediately that the row 1, column 3
value of the result must be like this:
\[
\begin{bmatrix}
\underline{a} & \underline{b} & \underline{c} \\
d & e & f 
\end{bmatrix}
\cdot
\begin{bmatrix}
A & B & \underline{C} & D \\
E & F & \underline{G} & H \\
I & J & \underline{K} & L  
\end{bmatrix}
= 
\begin{bmatrix}
? & ? & aC + bG + cK & ? \\
? & ? & ? & ?
\end{bmatrix}
\]
and row 1, column 4 of the result is
\[
\begin{bmatrix}
\underline{a} & \underline{b} & \underline{c} \\
d & e & f 
\end{bmatrix}
\cdot
\begin{bmatrix}
A & B & C & \underline{D} \\
E & F & G & \underline{H} \\
I & J & K & \underline{L}  
\end{bmatrix}
= 
\begin{bmatrix}
? & ? & ? & aD + bH + cL \\
? & ? & ? & ?
\end{bmatrix}
\]

Now what? 
We've been only moving the column on the second matrix on the left-hand-side.
And we've reached the end of that second matrix.
Now we move our eye on the row of the first matrix down and look at row 2.
While doing that, we move our eye on the last column of the second matrix
and look at the first column again.
And as for the resulting matrix, we jump to the second row.
This is what we get:
\[
\begin{bmatrix}
a & b & c \\
\underline{d} & \underline{e} & \underline{f} 
\end{bmatrix}
\cdot
\begin{bmatrix}
\underline{A} & B & C & D \\
\underline{E} & F & G & H \\
\underline{I} & J & K & L  
\end{bmatrix}
= 
\begin{bmatrix}
? & ? & ? & ?\\
dA + eE + fI & ? & ? & ?
\end{bmatrix}
\]
Now you move your eye on the second matrix from the 1st column to the second
and you'll get the row 2, column 2 value:
\[
\begin{bmatrix}
a & b & c \\
\underline{d} & \underline{e} & \underline{f} 
\end{bmatrix}
\cdot
\begin{bmatrix}
A & \underline{B} & C & D \\
E & \underline{F} & G & H \\
I & \underline{J} & K & L  
\end{bmatrix}
= 
\begin{bmatrix}
? & ? & ? & ?\\
? & dB + eF + fJ & ? & ?
\end{bmatrix}
\]
Next you get the row 2, column 3 of the resulting value like this:
\[
\begin{bmatrix}
a & b & c \\
\underline{d} & \underline{e} & \underline{f} 
\end{bmatrix}
\cdot
\begin{bmatrix}
A & B & \underline{C} & D \\
E & F & \underline{G} & H \\
I & J & \underline{K} & L  
\end{bmatrix}
= 
\begin{bmatrix}
? & ? & ? & ?\\
? & ? & dC + eG + fK & ?
\end{bmatrix}
\]
and finally this:
\[
\begin{bmatrix}
a & b & c \\
\underline{d} & \underline{e} & \underline{f} 
\end{bmatrix}
\cdot
\begin{bmatrix}
A & B & C & \underline{D} \\
E & F & G & \underline{H} \\
I & J & K & \underline{L}  
\end{bmatrix}
= 
\begin{bmatrix}
? & ? & ? & ?\\
? & ? & ? & dD + eH + fL 
\end{bmatrix}
\]

So if you want to compute that $(i,j)$ entry 
(i.e. entry at row $i$ and column $j$) of the product of
matrix $M$ and $N$, then you need to compute the 
dot product of row $i$ of $M$ with column $j$ of $N$.
If the column size of $M$ and row size of $N$ are both $s$,
and we write $M_{i,j}$ for the $(i,j)$ entry of $M$ and
$N_{i,j}$ for the entry of $N$, then the 
$(i,j)$ entry of $MN$ is
\[
M_{i,1}N_{1,j} + 
M_{i,2}N_{2,j} + \cdots +
M_{i,s}N_{s,j} 
\]

By the way, given two matrices $M$ and $N$,
if $MN$ and $NM$ are both defined, do you see that they must be
square matrix of the same size? 
(As the famous comedian Victor Borge use to say:
\lq\lq I'll give you a little more time on that one.'')

\newpage



Q4.
For those expressions where the operations do not make sense,
write ERROR.
Unless otherwise stated, all answers must be exact.

Let 
\[
A
= 
\begin{bmatrix}
1 & 2 & 3 \\
4 & 5 & 6 \\
\end{bmatrix}
, 
\hspace{1cm}
B
= 
\begin{bmatrix}
0.1 & -0.2 \\
-1.2 & 2.4 \\
-1.2 & 2.4 \\
\end{bmatrix}
, 
\hspace{1cm}
C
= 
\begin{bmatrix}
\frac{2}{3} & \frac{4}{7} \\
\end{bmatrix}
, 
\hspace{1cm}
D
= 
\begin{bmatrix}
3.14159 \\
\end{bmatrix}
\]
\[
E
= 
\begin{bmatrix}
0.33 & -1 & -2.5 \\
0.7 & 2.5 & 1.23
\end{bmatrix}
, 
\hspace{1cm}
F
= 
\begin{bmatrix}
2.7182 \\
\end{bmatrix}
, 
\hspace{1cm}
G
= 
\begin{bmatrix}
1.1 & -0.2 \\
1.2 & -0.4 \\
1.3 & -0.8 \\
\end{bmatrix}
, 
\hspace{1cm}
H
= 
\begin{bmatrix}
1 & 2 & -3 \\
3 & 2 & -1 \\
\end{bmatrix}
\]
\[
I
= 
\begin{bmatrix}
0.3 & 2.1 & -2.1 \\
0.2 & 2.5 & 3.7 \\
0.1 & 3.2 & 5.2 \\
\end{bmatrix}
, 
\hspace{1cm}
J
= 
\begin{bmatrix}
2.7182 \\
\end{bmatrix}
, 
\hspace{1cm}
K
= 
\begin{bmatrix}
120.5 \\
225.0 \\
174.4
\end{bmatrix}
, 
\hspace{1cm}
L
= 
\begin{bmatrix}
1 & 2 \\
-1 & -2 \\
\end{bmatrix}
\]

(a) Compute the size of $EB$.

(b) Compute the size of $IK$.

(c) Compute the size of $KI$.

(d) Compute the size of $CL$.

(e) Compute the size of matrix $X$ if you know that $IX$ 
is defined and has a size of 3-by-1.

(f) Compute $AB$.

(g) Compute $AE$.

(h) Compute $AG$.

(i) Compute $GE$.

(j) Compute $HGA$.

(k) Compute $DC$.

(l) Find two matrices (as simple as possible) $M$ and $N$ such that
$MN$ and $NM$ and are defined but $MN \neq NM$.
[HINT: A pair of 2-by-2 would do.]


\SOLUTION

(a)
$E$ is \underline{2}-by-3 and $B$ is 3-by-\underline{2}.
Therefore $EB$ is 2-by-2.
ANSWER: \boxed{\text{2-by-2}}

(b)
$I$ is \underline{3}-by-1 and $K$ is 3-by-\underline{1}.
Therefore $IK$ is 3-by-1.
ANSWER: \boxed{\text{3-by-1}}

(c)
ANSWER: \boxed{\text{ERROR}}

(d)
$C$ is \underline{1}-by-2 and $L$ is 2-by-\underline{2}.
Therefore $CL$ is 1-by-2
ANSWER: \boxed{\text{1-by-2}}

(e)

ANSWER: \boxed{\text{3-by-1}}


(f) 
\begin{align*}
AB &= \begin{bmatrix}
      1 & 2 & 3 \\
      4 & 5 & 6 \\
      \end{bmatrix}
      \cdot
      \begin{bmatrix}
      0.1 & -0.2 \\
      -1.2 & 2.4 \\
      -1.2 & 2.4 \\
      \end{bmatrix}
\\
   &= \begin{bmatrix}
      (1)(0.1) + (2)(-1.2) + (3)(-1.2) & (1)(-0.2) + (2)(2.4) + (3)(2.4) \\
       (4)(0.1) + (5)(-1.2) + (6)(-1.2) & (4)(-0.2) + (5)(2.4) + (6)(2.4) \\  
      \end{bmatrix}
\\
   &= \begin{bmatrix}
      -5.9 & 11.8 \\
      -12.8 & 25.6 \\  
      \end{bmatrix}
\end{align*}

ANSWER: 
\boxed{
\begin{bmatrix}
      -5.9 & 11.8 \\
      -12.8 & 25.6 \\  
      \end{bmatrix}
}

(g)
ANSWER: 
\boxed{ERROR}

(h)
\begin{align*}
AG &= \begin{bmatrix}
1 & 2 & 3 \\
4 & 5 & 6
\end{bmatrix}
\cdot
\begin{bmatrix}
1.1 & -0.2 \\
1.2 & -0.4 \\
1.3 & -0.8
\end{bmatrix} \\
&= \begin{bmatrix}
(1)(1.1) + (2)(1.2) + (3)(1.3) & (1)(-0.2) + (2)(-0.4) + (3)(-0.8)\\
(4)(1.1) + (5)(1.2) + (6)(1.3) & (4)(-0.2) + (5)(-0.4) + (6)(-0.8)
\end{bmatrix} \\
&= \begin{bmatrix}
7.4 & -3.4 \\
18.2 & -7.6
\end{bmatrix}
\end{align*}
ANSWER: \boxed{\begin{bmatrix}
7.4 & -3.4 \\
18.2 & -7.6
\end{bmatrix}}

(i)
\begin{align*}
GE &=
\begin{bmatrix}
1.1 & -0.2 \\
1.2 & -0.4 \\
1.3 & -0.8
\end{bmatrix}
\cdot
\begin{bmatrix}
0.33 &  -1 & -2.5 \\
0.7 & 2.5 & 1.23
\end{bmatrix} \\
&=
\begin{bmatrix}
(1.1)(0.33) + (1.2)(-1) + (1.3)(-2.5) & (1.1)(0.7) + (1.2)(2.5) + (1.3)(1.23) \\
(-0.2)(0.33) + (-0.4)(-1) + (-0.8)(-2.5) & (-0.2)(0.7) + (-0.4)(2.5) + (-0.8)(1.23)
\end{bmatrix} \\
&= \begin{bmatrix}
-4.087 & 5.369 \\
-2.334 & -2.124
\end{bmatrix}
\end{align*}
ANSWER: \boxed{\begin{bmatrix}
-4.087 & 5.369 \\
-2.334 & -2.124
\end{bmatrix}}

(j)
\begin{align*}
HGA &=
\begin{bmatrix}
1 & 2 & 3 \\
3 & 2 & -1
\end{bmatrix}
\cdot
\begin{bmatrix}
1.1 & -0.2 \\
1.2 & -0.4 \\
1.3 & -0.8
\end{bmatrix}
\cdot
\begin{bmatrix}
1 & 2 & 3 \\
4 & 5 & 6
\end{bmatrix} \\
&=
\begin{bmatrix}
(1)(1.1) + (2)(1.2) + (3)(1.3) & (1)(-0.2) + (2)(-0.4) + (3)(-0.8) \\
(3)(1.1) + (2)(1.2) + (-1)(1.3) & (3)(-0.2) + (2)(-0.4) + (-1)(-0.8) \\
\end{bmatrix}
\cdot
\begin{bmatrix}
1 & 2 & 3 \\
4 & 5 & 6
\end{bmatrix} \\
&=
\begin{bmatrix}
7.4 & -3.4 \\
4.4 & -0.6
\end{bmatrix}
\cdot
\begin{bmatrix}
1 & 2 & 3 \\
4 & 5 & 6
\end{bmatrix} \\
&=
\begin{bmatrix}
(7.4)(1) + (-3.4)(4) & (7.4)(2) + (-3.4)(5) & (7.4)(3) + (-3.4)(6) \\
(4.4)(1) + (-0.6)(4) & (4.4)(2) + (-0.6)(5) & (4.4)(3) + (-0.6)(6) \\
\end{bmatrix} \\
&=
\begin{bmatrix}
-6.2 & -2.2 & 1.8 \\
2 & 5.8 & 9.6
\end{bmatrix}
\end{align*}
ANSWER: \boxed{\begin{bmatrix}
-6.2 & -2.2 & 1.8 \\
2 & 5.8 & 9.6
\end{bmatrix}}

(k)
\begin{align*}
DC &=
\begin{bmatrix}
3.14159
\end{bmatrix}
\cdot
\begin{bmatrix}
\frac{2}{3} & \frac{4}{7}
\end{bmatrix} \\
&=
\begin{bmatrix}
(3.14159)(\frac{2}{3}) + (3.14159)(\frac{4}{7})
\end{bmatrix} \\
&=
\begin{bmatrix}
3.8896
\end{bmatrix}
\end{align*}
ANSWER: \boxed{\begin{bmatrix}
3.8896
\end{bmatrix}}

(l) 
Let 
\[
M = 
\begin{bmatrix}
1 & 1 \\
2 & 2 
\end{bmatrix}
\]
and
\[
N = 
\begin{bmatrix}
2 & 2 \\
1 & 1 
\end{bmatrix}
\]
Then
\begin{align*}
MN 
&= \begin{bmatrix} 
   1 & 1 \\
   2 & 2 
   \end{bmatrix}
   \cdot
   \begin{bmatrix}
   2 & 2 \\
   1 & 1 
   \end{bmatrix}
\\
&= \begin{bmatrix}
   3 & 3 \\
   6 & 6 
   \end{bmatrix}
\end{align*}
and
\begin{align*}
NM
&= \begin{bmatrix} 
   2 & 2 \\
   1 & 1 
   \end{bmatrix}
   \cdot
   \begin{bmatrix}
   1 & 1 \\
   2 & 2 
   \end{bmatrix}
\\
&= \begin{bmatrix}
   6 & 6 \\
   3 & 3 
   \end{bmatrix}
\end{align*}

ANSWER: 
\boxed{
M = \begin{pmatrix}
1 & 1 \\
2 & 2
\end{pmatrix},
\hspace{0.25cm}
N = \begin{pmatrix}
2 & 2 \\
1 & 1
\end{pmatrix}
}



\end{document}

